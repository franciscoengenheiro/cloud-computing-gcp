\chapter{Client Connection}\label{ch:client-connection}

The client connection is the part of the system that is responsible for connecting to the server.
As the server is a scalable service, the client needs to know the address of the server to connect to, which is only known dynamically.


\section{Cloud Http Function}\label{sec:cloud-http-functions}

A Google cloud HTTP function was developed to enable clients to discover all available server addresses dynamically.
This function fetches the IP addresses of running instances within a specified instance group
(as a query parameter) and returns them to the client in a single string format.
The format of the response is a simple string, where each address is separated by a semicolon (;).

The first implementation had a shuffling operation
(to enforce a way of load balancing) that was done on the client-side.
However, it was removed to allow the client to choose the server address to connect to from the list of addresses returned by the function.


\section{Client Lookup Service}\label{sec:client-lookup-service}

To connect to the server, a client lookup service was implemented.
This service uses a retry~\cite{retry-pattern} resilience mechanism to handle transient errors that may occur during the connection process.
This mechanism helps ensure that the client can connect to the server,
even if the server is not available at the time of the first connection attempts, as it could be overloaded or starting up.

According to the retry configuration, the client tries to connect to the server up to five times,
with a backoff factor strategy that doubles the waiting time between each retry.
Additionally,
the retry is set to retry on specific errors that occur during the connection process
(e.g., no ip address found, connection error, etc.).
Any other error that occurs during the connection process is not considered a transient error, and as such, the client lookup service does not retry.

The client lookup service is responsible
for fetching the server addresses from the previously mentioned Google Cloud HTTP function.
After a deserialization process, the client lookup service asks the client to choose an address to connect to.
With the selected address, the client lookup service tries to connect to the server,
and waits for a positive response of the connection status.
If the connection fails, the client lookup service retries the connection process as configured.

In later stages of the project, a problem occurred when the client lookup service was implemented.
If the server is a scalable service and the client is connected to a server instance that is being terminated,
the client would lose the connection and would not be able to reconnect to the server.
To solve this issue,
additional logic was added to the client code to handle this situation
and check if the server is still available upon each operation request.
If the server is not available, the client lookup service fetches the server addresses again
for the client to choose a new server address to connect to, if available.

Finally, a development mode was added to the client lookup service.
This mode was added to allow the client to connect to a local server instance for development purposes.
